\documentclass[twocolumn,twoside]{article}
\usepackage[english,spanish]{babel}
\usepackage[utf8]{inputenc}
\usepackage{indentfirst}
\usepackage{anysize} % Soporte para el comando \marginsize
%\marginsize{1.5cm}{1.5cm}{0.5cm}{1cm}
\marginsize{2,5cm}{1,8cm}{3.5cm}{1,7cm}
\usepackage[psamsfonts]{amssymb}
\usepackage{amssymb}
\usepackage{amsfonts}
\usepackage{amsmath}
\usepackage{amsthm}
\usepackage{placeins}
\usepackage{float}
\usepackage{varioref}
\usepackage{wrapfig}
\usepackage{ragged2e,tabularx,makecell}
\renewcommand{\thepage}{}
\columnsep=7mm

%%%%%%%%%%%%%%%%%%%%%%%%%%%%%%%%%%%%%%%%
\newtheorem{definicion}{Definici\'on}[section]
\newtheorem{teorema}{Teorema}[section]
\newtheorem{prueba}{Prueba}[section]
\newtheorem{prueba*}{Prueba}[section]
\newtheorem{corolario}{Corolario}[section]
\newtheorem{observacion}{Observaci\'on}[section]
\newtheorem{lema}{Lema}[section]
\newtheorem{ejemplo}{Ejemplo}[section]
\newtheorem{solucion*}{Soluci\'on}[section]
\newtheorem{algoritmo}{Algoritmo}[section]
\newtheorem{proposicion}{Proposici\'on}[section]

\linespread{1.4} \sloppy

\newcommand{\R}{\mathbf{R}}
\newcommand{\N}{\mathbf{N}}
\newcommand{\C}{\mathbb{C}}
\newcommand{\Lr}{\mathcal{L}}
\newcommand{\fc}{\displaystyle\frac}
\newcommand{\ds}{\displaystyle}

\DeclareMathOperator{\Dom}{Dom}

%%%%%%%%%%%%%%%%%%%%%%%%%%%%%%%%%%%%%%%%

\renewcommand{\thefootnote}{\fnsymbol{footnote}}

\begin{document}
\begin{center}
 {\Large \textbf{Aplicaci\'on del algoritmo de transformaci\'on de Householder en m\'inimos cuadrados para un an\'alisis 
 de la pobreza en el Per\'u}}
\end{center}
\begin{center}
 Ayrton Fabio Coronado Huam\'an $^1$, Guillermo Joel Borjas C\'ordova $^2$,
 Israel Danilo Blas Salas $^3$, Tom\'as Garc\'ia Sifuentes $^4$ \vskip12pt
{\it Facultad de Ciencias $1$, Universidad Nacional de Ingenier\'{\i}a $1$, e-mail: \\
Facultad de Ciencias $2$, Universidad Nacional de Ingenier\'{\i}a $2$, e-mail:  }
\end{center}
%\maketitle 
\begin{quotation}
{\small
%\begin{abstract}

%\end{abstract}
Palabras Claves:  
}\\
{\small
\hspace*{0.5cm} 

Keywords: \\ 
}
\end{quotation}

\section{Introducc\'ion}

En este proyecto nos dedicaremos a construir un algoritmo eficiente, en t\'erminos de
tiempo de ejecuci\'on, del m\'etodo matemático "Transformación de Householder", el cual
nos permite obtener un tipo de descomposici\'on de matrices llamado "Descomposici\'on QR".
Esto nos servir\'a en la resoluci\'on de un sistema de ecuaciones proveniente del ajuste de
una curva utilizando la t\'ecnica de m\'inimos cuadrados donde intervienen una variable 
dependiente y una independiente, y en la cual la relaci\'on entre ellas se aproxima por medio
de una l\'inea recta, todo esto con la finalidad de poder aplicarse a una situaci\'on 
particular en el campo de la economía, m\'as concretamente, para hacer un an\'alisis de la 
pobreza en el Per\'u y de esta manera hacer predicciones o pronósticos que nos ayudarán a 
tener un mejor entendimiento de este problema.\\

Con el fin de llevar a cabo este proyecto, se emplearán los datos tomados de una muestra real, 
extra\'idos de un informe hecha por el INEI de pobreza, pobreza extrema y el coeficiente de
Gini (indicador de la desigualdad econ\'omica en una poblaci\'on) en los años 2010 - 2013 de los
principales departamentos del Per\'u.\\

Muchos autores que han hecho estudios sobre
modelos de regresi\'on, entre los que se pueden
citar a: Anderson, D. R., Sweeney, D. J., \& Williams,
T. A. (2001), Devore, J. L. (2005), Evans, M., \&
Rosenthal, J. S. (2005), Freund, J. E., \& Simon,
G. A. (1994), Levin, R. I., \& Rubin, D. S. (2004)
y Miller, I. (2000); coinciden en que siempre que
se analizan datos observados o recopilados para 
llegar a una funci\'on o ecuaci\'on matem\'atica que
describa la relaci\'on entre las variables por medio de
una regresi\'on, se deben enfrentar tres problemas:\\
1. Decidir qu\'e clase de curva muestran los puntos
y por tanto qu\'e clase de ecuaci\'on se debe usar.\\
2. Encontrar la ecuaci\'on particular que mejor se
ajuste a los datos.\\
3. Demostrar que la ecuaci\'on particular encontrada
cumple con ciertos aspectos referentes a los méritos 
de \'esta para hacer pron\'osticos.\\

Para decidir qu\'e clase de funci\'on podr\'ia ajustarse
a la curva, debe hacerse una gr\'afica de dispersi\'on
de los datos observados. Si en dicha gr\'afica se aprecia 
que los puntos se distribuyen alrededor de una recta, se 
procede a realizar un an\'alisis de regresi\'on lineal.


\section{Conceptos Previos}
Al igual que en el problema del an\'alisis de pobreza, a menudo coleccionamos datos e 
intentamos encontrar una relación funcional entre las variables. Si los datos son $n+1$ 
puntos del plano, es posible encontrar un polinomio de grado $n$ o inferior que pasa por 
todos los puntos. Este polinomio se llama polinomio de interpolación.
Dado que los datos tienen en general error experimental,
no hay razón para pedir que las funciones pasen por todos los puntos. De hecho, polinomios
de grado inferior que no pasan por los puntos de manera exacta, dan una mejor
descripción de la relación entre variables. Por ejemplo, si la relación entre variables es
actualmente lineal y los datos tienen un pequeño error, ser\'ia desastroso usar un polinomio
de interpolación.\\
Dada una tabla de datos \\
  \
  \
    \begin{table}[!htb]
    \begin{center}
        
    
    \begin{tabular}{|l|l|l|l|l|}
    \hline
    
    $x$&   $x_{1}$&   $x_{2}$&   $...$&  $x_{m}$ \\ \hline
    $y$&   $y_{1}$&   $y_{2}$&   $...$&  $x_{m}$ \\ 
    \hline 
    \end{tabular}
    
    \end{center}
    \end{table} 
\
\
\

deseamos encontrar una función lineal

%%%formula

que mejor aproxima los datos en el sentido de m\'inimos cuadrados. Si se pide que
%formula     para     %formula    
obtenemos un sistema de $m$ ecuaciones en dos incognitas
%formula matrices
La función lineal cuyos coeficientes son la solución de m\'inimos cuadrados de %ecuacion
viene a ser la aproximación de m\'inimos cuadrados a los datos con una funci\'on lineal. \\
Si los datos no aparecen en relación lineal, se podria usar un polinomio de grado
mayor. Es decir, para encontrar los coeficientes $c_{0}, c{1}, . . ., c_{n}$ de la mejor aproximación
por m\'inimos cuadrados a los datos 
%tabla
con un polinomio de grado $n$, tenemos que encontrar la solución de m\'inimos cuadrados
al sistema
%matriz
\subsection{LAS TRANSFORMACIONES DE HOUSEHOLDER}
Debido a las posibles dificultades numéricas del uso de las ecuaciones normales, los
modernos métodos de m\'inimos cuadrados se han desarrollado basándose en las transformaciones 
ortogonales, que preservan las distancias eucl\'ideas y no empeoran las
condiciones de la matriz A. La idea es transformar un problema de m\'inimos cuadrados
de manera que sea fácil de resolver, reduciendo la matriz A a la forma que revela el rango.
El término forma triangular que revela el rango una matriz genérica $m\times n$ de rango $r$
correspondiente a la matriz $ T' = $%% matrices
,con $T_{11}$ matriz $r x r$ no singular triangular superior y $T_{12}$ matriz $r\times (n-r)$.
Cuando$ T' $ tiene rango lleno de filar entonces no aparecen ambos bloques de ceros; si $T'$ es de rango lleno de columna, la
matriz $T_{12}$ y el bloque de ceros derecho no aparecerán. Mas en general, el rango de una 
matriz $m\times n$, $F'$ , es $r$ si $F'$ es de la forma $F' =$%matriz
, con $F$ matriz $r\times n$ y las filas de $F$ son linealmente independientes.
Dado que ya no estamos resolviendo ecuaciones, el conjunto de transformaciones
que se puede aplicar a $A$ sin cambiar la solución está restringido. Las transformaciones
ortogonales son obvias en este contexto dado que estas no alteran la norma $l_{2}$.
Sea A una matriz no nula $m\times n$, con rango igual a $r$. Supóngase que se pueda
encontrar una matriz $m\times m$ ortogonal $Q$ que produzca la forma que revela el rango
%formula
donde $F$ es una matriz $r\times n$ y tiene las filas linealmente independientes; el bloque de
ceros no aparece si $r = m$.
La forma (XX.1) nos permite resolver el problema de m\'inimos cuadrados usando las
matrices $Q$ y $F$ . Sea d el vector transformado $Q^{t} b$, descompuesto en
%formula  (XX.2)
Usando esta definición y la estructura mostrada en (XX.1), se puede escribir el vector
residual transformado como
%formula (XX.3)
Dado que la matriz $Q^{t}$ es ortogonal, su aplicación al vector residual no altera la norma
eucl\'idea, y
%formula  (XX.4)
Combinando esta relación con la forma especial del vector residual transformado, se concluye que
%formula(XX.5)
que significa que $||b-Ax||_{2}$  no puede ser menor que $||d_{m-r} ||$ 2 para cualquier vector x.
El valor más pequeño posible para $||b-Ax||_{2}^{2}$ es la cota inferior. Igualdades con la cota
inferior se obtienen si y sólo si el vector $x$ satisface
%%formula(XX.6)
Dado que el rango de la matriz $F$ es igual a $r$, el sistema $F x = d_r$ tiene que ser compatible.
Cuando $F x = d_r$ , el vector residual transformado satisface $Q^t$ %formula
y $||b - A x||_2 = ||d m-r ||_2$   . Esto demuestra que cualquier vector $x$ que satisfaga $F x = d_r$
será una solución de m\'inimos cuadrados. Suponiendo que los sistemas en los cuales
aparecen las matrices F son fáciles de resolver, se sigue que el problema de m\'inimos
cuadrados se puede resolver encontrando una matriz ortogonal $Q_t$ que nos dé la forma
(XX.1).
Antes de presentar la factorización QR para resolver el problema de m\'inimos
cuadrados, es necesario introducir las transformaciones de Householder.\\
La técnica más popular para construir una matriz ortogonal que reduzca la matriz $A$
en forma triangular usa una clase especial de matrices que son simultaneamente
simétricas, elementales y ortogonales. Para cualquier vector no nulo u, la correspondiente 
transformación de Householder (o matriz de Householder, o reflector de
Householder) es una matriz de la forma
%formula (XX.7)
donde el vector $u$ es el vector de Householder
Teorema XX.1
Si H es la matriz definida en (XX.7), entonces
%listar
1) $H = H^t$ ,
2) $H = H^1$ ,
que es lo mismo que decir que la matriz H es simétrica y ortogonal.

Entonces, las matrices de Householder son simétricas y ortogonales, y dependen sólo
de la dirección del vector $u$.
En el contexto de las reducciones a matrices triangulares, las matrices de Householder
poseen dos propiedades cruciales:
%listar
- para cualquier par de vectores distintos de igual norma $l_2$ , existe una transformación de
Householder que transforma el uno en el otro,
%formula (XX.8)
con $||a||_2 = ||b||_2$ . Esto implica que el vector $u$ tiene que satisfacer la condición
%formula (XX.9)
es decir, $u$ es un múltiplo de $b-a$;
- cualquier vector c transformado por una matriz de Householder posee una forma especial:
%formula (XX.10)
de manera que $H c$ es la diferencia entre el vector original $c$ y un múltiplo especial del
vector de Householder $u$.
Claramente el vector $c$ no var\'ia si $u^t c = 0$. Además, calcular el producto $H c$ no necesita
los elementos expl\'icitos de $H$, sino sólo el vector $u$ y el escalar $\beta$.
\subsection{LA FACTORIZACI\'ON QR}
Para una matriz genérica $A$, $n\times n$, no singular, las propiedades que se acaban de
describir permiten construir una sucesión de $n-1$ matrices de Householder tales que
%formula (XX.11)
donde $R$ es una matriz $n\times n$ no singular y triangular superior.
El primer paso de este proceso es construir una matriz de Householder $H_1$ que
transforma $a_1$ (la primera columna de $A$) en un múltiplo de $e_1$ , es decir, se desean crear
ceros en las componentes $2$ hasta $n$ del vector $a_1$ . La norma eucl\'idea se conserva bajo
transformaciones ortogonales, de manera que
%formula (XX.12)
donde $|r_{11} | = ||a_1||_2$ . De la expresión (XX.9) sabemos que el vector $u_1$ tiene que ser un
múltiplo de $||a_1||_2e_1-a_1$ , y dado que $H_1$ depende sólo de la dirección de $u_1$ , podemos
elegir el vector $u_1$ como
%formula matriz(XX.13)
Al definir $u_1$ , el signo de $r_{11}$ se puede eligir positivo o negativo (excepto cuando $a_1$ es
ya un múltiplo de $e_1$ ), y para evitar el problema de la cancelación de términos parecidos,
usualmente se escoge el signo opuesto al signo de $a_11$ , de manera que
%formula (XX.14)
Después de la primera aplicación de las transformaciones de Householder, la primera
columna de la matriz parcialmente reducida $A^{(2)}=H_1 A $es un múltiplo de $e_1$ , y los
demás elementos han sido alterados
%formula (XX.15)
En muy importante notar que a diferencia de la eliminación Gaussiana, la primera fila de
la matriz $A$ viene modificada por efecto de la transformación de Householder $H$.
Al construir la segunda transformación de Householder, el objetivo principal es reducir
la segunda columna a la forma correcta, sin alterar la primera fila y la primera
columna de la matriz parcialmente reducida. Debido a la propiedad (XX.10), se puede
obtener este resultado definiendo el segundo vector de Householder $u_2$ con la primera
componente nula. Habiendo escogido as\'i el vector $u_2$ , la aplicación de la matriz de
Householde $H_2$ a un vector genérico no cambia la primera componente, y la aplicación a
un múltiplo de $e_1$ deja el vector entero como está.
Si $A$ es una matriz no singular, se pueden efectuar $n-1$ pasos de reducción de
Householder, para obtener $H_{n-1} . . . H_2 H_1 A = R$, con $R$ una matriz triangular superior
no singular. Si se denota con $Q^t$ la matriz ortogonal $n\times n$
%formula (XX.16)
Cualquiera de las dos formas
%formula (XX.17)
se conoce como la factorización QR de la matriz $A$.
Una vez conocida la factorización QR de $A$, ecuación (XX.17), la solución al sistema
$A x = b$ se obtiene resolviendo el sistema triangular superior $R x = Q^t b$.
En general, es necesario un intercambio de columnas para asegurar que el rango
está plenamente revelado. Sin el intercambio de columnas la reducción de Householder
terminar\'ia inmediatamente con una matriz no nula cuya primera columna es cero. Si las
demás columnas son también ceros, la reducción termina. De otra manera, existe por
lo menos una columna no nula, la columna pivote, que se puede eligir como candidata
para la siguiente reducción. Como en la eliminación gaussiana, una estrategia de pivoteo
pide que se escoja la columna pivote como la primera columna de norma máxima (otra
posibilidad es escoger la columna “menos reducida”).
En general, si $A$ es $m\times n$ de rango $r$, se necesitarán $r$ permutaciones ${P_k}$ y $r$
matrices de Householder ${H_k}$, $k = 1, . . . , r$. Después de estos $r$ pasos, la configuración
final será
%formula(XX.18)
donde
%formula(XX.19)
es triangular superior que nos revela el rango, y $R_{11}$ es una matriz $ r\times r$ no singular
triangular superior. Con una correcta estrategia de pivoteo, dependiente del problema
original, y aritmética exacta, este procedimiento de Householder terminará después de $r$
pasos, cuando la matriz restante se transformará en cero. Combinando los intercambios
de columnas en una sola matriz de permutación $P$ y las transformaciones de Householder
en una sola matriz ortogonal $Q^t$ , se obtiene
%formula(XX.20)
o equivalentemente,
%formula(XX.21)
y 
%formula(XX.22)
Las filas de la matriz $R P^t$ son linealmente independientes, de manera que la matriz
transformada $Q^t A$ tiene la forma deseada (XX.1), con $F = R P^t$ . El problema de
m\'inimos cuadrados se puede entonces resolver con el siguiente algoritmo:
%enumerar
(i) Calcular la factorización QR de la matriz $A$, usando la reducción de Householder (que
determina el rango $r$, y las matrices $P$ , $Q$ y $R$);
(ii) Formar el vector $d = Q^t b$, y denotar con $d_r$ las primeras $r$ componentes de $d$;
(iii) Calcular cualquier solución $y$ del sistema $R_y = d_r$ ;
(iv) Formar $x = Py$.
El número de operaciones requerido para resolver el problema de m\'inimos cuadrados
de esta manera es del orden de $2\cdot m\cdot n\cdot r - r^2 (m + n) + \frac{2}{3} n^3$ .
Si la matriz $A$ posee columnas linealmente independientes $(r = n)$, la matriz $R$ es no
singular, la solución de $Ry = d_r$ es única, y la solución del problema de m\'inimos cuadrados 
es única. En este caso resolver el problema con el método QR es aproximadamente
dos veces más costoso que resolviendolo con las ecuaciones normales. Cuando las columnas
 de A son linealmente dependientes, de manera que $r < n$, el sistema $r\times n$ $Ry = d_r$
posee un número infinito de soluciones. Dado que $R$ es de la forma $R = (R_{11} R_{12} )$, con
$R_{11}$ triangular superior no singular, es posible escoger un vector $y = (y_B , 0)^t$ tal que
R y = d r y con la propiedad de que R 11 y B = d r . Si y posee esta forma, la solución
$x = Py$ es sencillamente una reordenación de $y$, y es llamada una solución básica del
problema de m\'inimos cuadrados.
\begin{center}
{\large \bf 3. AN\'ALISIS}
\end{center}

\begin{center}
{\large \bf 4. OBSERVACIONES}
\end{center}

\begin{center}
{\large \bf 5. CONCLUSIONES}
\end{center}

%\begin{center}
%{\large \bf Agradecimientos}
%\end{center}
%Los autores agradecen a las autoridades de la Facultad de Ciencias de la Universidad Nacional de 
%Ingenier\'{\i}a por su apoyo.
%%\begin{center}
%%{\large \bf Apendice: }
%%\end{center}



\begin{center}
 -----------------------------------------------------------------------------
\end{center}

\begin{list}{}{\setlength{\topsep}{0mm}\setlength{\itemsep}{0mm}%
\setlength{\parsep}{0mm}\setlength{\leftmargin}{4mm}}
%
%------------------------------------- References --------------------
\small
\item[1.] I.K. Argyros, \textit{Newton-like methods under mild \linebreak differentiability conditions with error analysis,} Bull. \linebreak Austral. Math. Soc. \textbf{37} (1988), 131-147.
\item[2.] 
%---------------------------------------------------------------------
%
\end{list}
\end{document}\grid
